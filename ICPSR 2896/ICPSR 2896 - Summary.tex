\documentclass[12pt]{report}
\usepackage[utf8]{inputenc}
\usepackage[T1]{fontenc}
\usepackage{textgreek}
\usepackage{titling}
\usepackage{fullpage}
\usepackage{bold-extra}

\newcommand{\subtitle}[1]{%
	\posttitle{%
		\par\end{center}
		\begin{center}\large#1\end{center}
		\vskip0.5em}%
}

\title{\Huge \bf Document Summary}
\subtitle{ICPSR 2896 --- \textit{Historical, Demographic, Economic, and Social Data: The United States, 1790--2002}}
\date{June 13, 2014}
\author{Jade Coppieters}

\begin{document}

\maketitle
\tableofcontents
\newpage

\chapter{Summary of Select Datasets, 1850--1920}
\newpage

\section{DS7 --- 1850 Census}
	Data Level: State and National\\*
	Number of Variables (Columns): 313\\*
	Number of Entries (Rows): 1671
	\vspace{1.5em}
	
	\noindent Quick Summary: Most of the variables in this dataset are demographic in nature. Important variables include educational, agricultural, and religious data, as well as literacy rates. \textit{There is no data specifically concerning banks or the financial sector.}
	
	\paragraph{Data Specifics} (\guillemotleft~\guillemotright~indicates a break in data information, consult codebook):
	\vspace{1em}
	
	\begin{tabular}{r|l}
		001--003 & Locational Data\\
		004--006 & Population Data\\
		007--039, 182 & Demographic Data for White Population\\
		040--072 & Demographic Data for Free Colored Population\\
		073--105 & Demographic Data for Slave Population\\
		106--113 & Vital Statistics Data; Births Marriages, Deaths\\
		114--115 & \guillemotleft~\guillemotright\\
		116--123 & College Data\\
		124 & \guillemotleft~\guillemotright\\
		125--131 & Public School Data\\
		132--139 & Academy Data\\
		140--148, 181 & Educational Data\\
		149--157 & Literacy Rates\\
		157--166, 183 & Agricultural Data\\
		167--178 & Library Data\\
		179--180 & \guillemotleft~\guillemotright\\
		184--186 & Manufacturing Data\\
		187--258 & Religious Data\\
		259--260 & Transportation Data\\
		261--276 & Demographic Data for Males\\
		277--292 & Demographic Data for Females\\
		293--308 & Demographic Data for Total Population\\
		309--318 & \guillemotleft~\guillemotright
	\end{tabular}
	\newpage

\section{DS8 --- 1850 Census}
	Data Level: State\\*
	Number of Variables (Columns): 200\\*
	Number of Entries (Rows): 37
	\vspace{1.5em}
	
	\noindent Quick Summary: Offers more significant, unique data than DS7. Aside from demographic information, this dataset contains data on slaveholders, labor composition, wages, manufacturing, capital, agriculture, and publishing. \textit{There \emph{\textbf{is}} data specifically concerning banks or the financial sector in this dataset.}
	
	\paragraph{Data Specifics} (\guillemotleft~\guillemotright~indicates a break in data information, consult codebook):
	\vspace{1em}
	
	\begin{tabular}{r|l}
		001--002 & Locational Data\\
		003--014 & Demographic Data for White Population\\
		015--026 & Demographic Data for Free Colored Population\\
		027--060 & Population Data by State\\
		061--093 & Population Data by Nation\\
		094--096 & \guillemotleft~\guillemotright\\
		097--108 & Slave Data\\
		109--119 & Industry-Labor Composition Data\\
		120--125 & Wage Data\\
		\textbf{126--132} & \textsc{\textbf{Bank and Financial Data}}\\
		133--147 & Manufacturing Data\\
		148--152 & Agricultural Data\\
		153--184 & Newspaper Data\\
		185--196 & Library Data\\
		197--200 & \guillemotleft~\guillemotright\\
	\end{tabular}
	\newpage

\section{DS9 --- 1860 Census}
	Data Level: State and National\\*
	Number of Variables (Columns): 427\\*
	Number of Entries (Rows): 2149
	\vspace{1.5em}
	
	\noindent Quick Summary: Still mainly focused on demographics, with supplemental data on agriculture, slaveholders, manufacturing, and religion. \textit{There is no data specifically concerning banks or the financial sector.}
	
	\paragraph{Data Specifics} (\guillemotleft~\guillemotright~indicates a break in data information, consult codebook):
	\vspace{1em}
	
	\begin{tabular}{r|l}
		001--003 & Locational Data\\
		004--006 & Population Data\\
		007--039 & Demographic Data for White Population\\
		040--072 & Demographic Data for Free Colored Population\\
		073--105 & Demographic Data for Slave Population\\
		106--138 & Demographic Data for Indian Population\\
		139--159 & Demographic Data\\
		160--175 & Agricultural Data\\
		176--198 & Slave Data\\
		199--205 & Manufacturing Data\\
		206--209 & \guillemotleft~\guillemotright\\
		210--305 & Religious Data\\
		306--338 & Demographic Data for "Half-Breed" Population\\
		339--371 & Demographic Data for Asian Population\\
		372--373 & Transportation Data\\
		374--389, 421 & Demographic Data for Males\\
		390--405, 422 & Demographic Data for Females\\
		406--420 & Demographic Data for Total Population\\
		423--427 & \guillemotleft~\guillemotright\\
	\end{tabular}
	\newpage

\section{DS10 --- 1860 Census}
	Data Level: State\\*
	Number of Variables (Columns): 245\\*
	Number of Entries (Rows): 43
	\vspace{1.5em}

	\noindent Quick Summary: Datasets are beginning to broaden, with similar data to DS8. \textit{There \emph{\textbf{is}} data specifically concerning banks or the financial sector in this dataset.}
	
	\paragraph{Data Specifics} (\guillemotleft~\guillemotright~indicates a break in data information, consult codebook):
	\vspace{1em}
	
	\begin{tabular}{r|l}
		001--002 & Locational Data\\
		003--041 & Population Data by State\\
		042--085 & Population Data by Nation\\
		086--103 & \guillemotleft~\guillemotright\\
		104--109 & Wage Data\\
		110--120 & Tax Data\\
		\textbf{121--129} & \textsc{\textbf{Bank and Financial Data}}\\
		130--140 & Transportation Data\\
		141--144 & Agricultural Data\\
		145--179 & Newspaper Data\\
		180--188, 198--205 & Educational Data\\
		189--197 & Literacy Rates\\
		206--217 & Library Data\\
		218--225 & College Data\\
		226--233 & Public School Data\\
		234--241 & Academy Data\\
		242--245 & \guillemotleft~\guillemotright\\
	\end{tabular}
	\newpage

\section{DS11 --- 1870 Census}
	Data Level: State\\*
	Number of Variables (Columns): 216\\*
	Number of Entries (Rows): 2365
	\vspace{1.5em}
	
	\noindent Quick Summary: Expanded but similar to DS9. \textit{There is no data specifically concerning banks but there \emph{\textbf{is}} data concerning public debt.}
	
	\paragraph{Data Specifics} (\guillemotleft~\guillemotright~indicates a break in data information, consult codebook):
	\vspace{1em}
	
	\begin{tabular}{r|l}
		001--003 & Locational Data\\
		004--017, 124--131 & Population Data\\
		018--067 & Population Data by State\\
		068--095 & Population Data by Nation\\
		096--105 & Educational Data\\
		106--123 & Literacy Rate\\
		132--153 & Agricultural Data\\
		154--164 & Manufacturing Data\\
		165--168 & Tax Data\\
		\textbf{169--170} & \textsc{\textbf{Public Data}}\\
		171--212 & Religious Data\\
		213--216 & \guillemotleft~\guillemotright\\
	\end{tabular}
	\newpage

\section{DS12 --- 1870 Census}
	Data Level: State\\*
	Number of Variables (Columns): 601\\*
	Number of Entries (Rows): 48
	\vspace{1.5em}
	
	\noindent Quick Summary: This dataset does not contain significant variables but does contain dramatically expanded demographic data. Compare DS12, DS13, DS14 as a set. \textit{There is no data specifically concerning banks or the financial sector.}
	
	\paragraph{Data Specifics} (\guillemotleft~\guillemotright~indicates a break in data information, consult codebook):
	\vspace{1em}
	
	\begin{tabular}{r|l}
		001--002 & Locational Data\\
		003--053 & Population Data by State\\
		054--121 & Population Data by Nation\\
		122--171 & Population Data for White Population by State\\
		172--240 & Population Data for White Population by Nation\\
		241--291 & Population Data for Colored Population by State\\
		292--359 & Population Data for Colored Population by Nation\\
		360--410 & Population Data for Indian Population by State\\
		411--478 & Population Data for Indian Population by Nation\\
		479--529 & Population Data for Chinese Population by State\\
		530--597 & Population Data for Chinese Population by Nation\\
		598--601 & \guillemotleft~\guillemotright\\
	\end{tabular}
	\newpage

\section{DS13 --- 1870 Census}
	Data Level: State\\*
	Number of Variables (Columns): 229\\*
	Number of Entries (Rows): 48
	\vspace{1.5em}
	
	\noindent Quick Summary: This dataset has data on religious, public debt, and tax variables, as well as demographic data subdivided by industry. Compare DS12, DS13, DS14 as a set. \textit{There is no data specifically concerning banks or the financial sector.}
	
	\paragraph{Data Specifics} (\guillemotleft~\guillemotright~indicates a break in data information, consult codebook):
	\vspace{1em}
	
	\begin{tabular}{r|l}
		001--002 & Locational Data\\
		003--114 & Religious Data\\
		115--134 & Demographic Data for Employed Population\\
		135--154 & Demographic Data for Employed Population, Agriculture\\
		155--174 & Demographic Data for Employed Population, Professional Service\\
		175--194 & Demographic Data for Employed Population, Transport\\
		195--214 & \parbox[t]{10cm}{Demographic Data for Employed Population,\\ Manufacturing/Mechanical Industries/Mining}\\
		215--221 & Public Data\\
		222--225 & Tax data\\
		226--229 & \guillemotleft~\guillemotright
	\end{tabular}
	\newpage

\section{DS14 --- 1870 Census}
	Data Level: State\\*
	Number of Variables (Columns): 289\\*
	Number of Entries (Rows): 48
	\vspace{1.5em}
	
	\noindent Quick Summary: This dataset contains expanded data on education and demographics. Compare DS12, DS13, DS14 as a set. \textit{There is no data specifically concerning banks or the financial sector.}
	
	\paragraph{Data Specifics} (\guillemotleft~\guillemotright~indicates a break in data information, consult codebook):
	\vspace{1em}
	
	\begin{tabular}{r|l}
		001--002 & Locational Data\\
		003 & Population Data\\
		004--014 & Educational Data\\
		015--025 & Public School Data\\
		026--036 & Classical/Professional/Technical (CPT) School Data\\
		037--047 & Other Non-Public (ONP) School Data\\
		048--051 & Educational Data by Race\\
		052--063 & Literacy Rates\\
		064--089 & Library Data\\
		090--095 & \guillemotleft~\guillemotright\\
		096--097 & Population Data by Gender\\
		098--124 & Population Data by Race and Age\\
		125--126 & Vital Statistics Data, Deaths\\
		127--130 & \guillemotleft~\guillemotright\\
		131--285 & Newspaper Data\\
		286--289 & \guillemotleft~\guillemotright
	\end{tabular}
	\newpage

\section{DS15 --- 1880 Census}
	Data Level: State and National\\*
	Number of Variables (Columns): 188\\*
	Number of Entries (Rows): 2663
	\vspace{1.5em}
	
	\noindent Quick Summary: Aside from data included in previous datasets, the agricultural data in this set distinguishes between owned farms, tenant farms, and share-cropped farms. \textit{There is no data specifically concerning banks or the financial sector.}
	
	\paragraph{Data Specifics} (\guillemotleft~\guillemotright~indicates a break in data information, consult codebook):
	\vspace{1em}
	
	\begin{tabular}{r|l}
		001--003 & Locational Data\\
		004--016 & Population Data\\
		017--039 & Manufacturing Data\\
		040--089, 095--100 & Agricultural Data\\
		090--094 & Transportation Data\\
		101--102 & \guillemotleft~\guillemotright\\
		103--105 & Tax Data\\
		106 & \guillemotleft~\guillemotright\\
		107--132 & Population Data by Nation\\
		133--186 & Population Data by State\\
		187--188 & \guillemotleft~\guillemotright
	\end{tabular}
	\newpage

\section{DS16 --- 1880 Census}
	Data Level: State\\*
	Number of Variables (Columns): 397\\*
	Number of Entries (Rows): 48
	\vspace{1.5em}
	
	\noindent Quick Summary: DS16 is an expanded dataset focusing on population and demographics, with specific subset information on manufacturing industries. Compare DS16 and DS17 as a set. \textit{There \emph{\textbf{is}} data specifically concerning banks or the financial sector in this dataset.}
	
	\paragraph{Data Specifics} (\guillemotleft~\guillemotright~indicates a break in data information, consult codebook):
	\vspace{1em}
	
	\begin{tabular}{r|l}
		001--002 & Locational Data\\
		003--052 & Population Data for White Population by State\\
		053--102 & Population Data for Colored Population by State\\
		103--168 & Population Data by Nation\\
		169--184 & Population Data\\
		185--215 & Educational Data\\
		216--290 & Newspaper Data\\
		291 & Transportation Data\\
		\textbf{292--314} & \textsc{ \textbf{Bond Data} }\\
		315--329 & \parbox[t]{10cm}{Demographic Data for Employed Population,\\ Manufacturing/Mining}\\
		330--334, 350--379 & Manufacturing Data, Cotton Industry\\
		335--349 & Demographic Data for Employed Population, Cotton Industry\\
		380--382, 390--393 & Manufacturing Data, Iron/Steel Industry\\
		383--389 & \parbox[t]{10cm}{Demographic Data for Employed Population,\\ Iron/Steel Industry}\\
		394--396 & Population Data by State\\
		397 & \guillemotleft~\guillemotright
	\end{tabular}
	\newpage

\section{DS17 --- 1880 Census}
	Data Level: State\\*
	Number of Variables (Columns): 484\\*
	Number of Entries (Rows): 48
	\vspace{1.5em}
	
	\noindent Quick Summary: This dataset includes the most comprehensive collection of demographic data thus far. Compare DS16 and DS17 as a set. \textit{There is no data specifically concerning banks or the financial sector.}
	
	\paragraph{Data Specifics} (\guillemotleft~\guillemotright~indicates a break in data information, consult codebook):
	\vspace{1em}
	
	\begin{tabular}{r|l}
		001--002 & Locational Data\\
		003--031, 042 & Population Data\\
		035--041 & Vital Statistics Data, Deaths\\
		043--071 & Literacy Rates\\
		072--077 & Transportation Data\\
		078--093 & Agricultural Data\\
		094--109 & Population Data for Employed Population, Shoe Industry\\
		110--125 & Population Data for Employed Population, Clerk/Assistant Industry\\
		126--141 & Population Data for Employed Population, Clergy\\
		142--157 & Population Data for Employed Population, Textile Manufacturing\\
		158--173 & \parbox[t]{10cm}{Population Data for Employed Population,\\ Domestic Servant Population}\\
		174--189 & Population Data for Employed Population, Agriculture\\
		190--205 & Population Data for Employed Population, Iron/Steel Industry\\
		206--221 & Population Data for Employed Population, Laborers\\
		222--237 & Population Data for Employed Population, Lawyers\\
		238--253 & Population Data for Employed Population, Officials\\
		254--269 & Population Data for Employed Population, Mining\\
		270--285 & Population Data for Employed Population, Government Service\\
		286--301 & Population Data for Employed Population, Railroad\\
		302--317 & Population Data for Employed Population, Physicians/Surgeons\\
		318--333 & Population Data for Employed Population, Mariners\\
		334--349 & Population Data for Employed Population, Drovers/Herders\\
		350--365 & Population Data for Employed Population, Tailors\\
		366--381 & Population Data for Employed Population, Teachers\\
		382--397 & Population Data for Employed Population, Traders and Dealers\\
		398--406 & Population Data by Gender and Age\\
		407--413 & Population Data by Nation
	\end{tabular}
	\newpage
	\begin{tabular}{r|l}
		414--429 & Demographic Data for Employed Population, Agriculture\\
		430--445 & Demographic Data for Employed Population, Professional Service\\
		446--461 & Demographic Data for Employed Population, Trade/Transport\\
		462--477 & \parbox[t]{10cm}{Demographic Data for Employed Population,\\ Manufacturing/Mechanical Industry/Mining}\\
		478--480 & Agricultural Data\\
		481--483 & Population Data by State\\
		484 & \guillemotleft~\guillemotright
	\end{tabular}
	\newpage

\section{DS18 --- 1890 Census}
	Data Level: State and National\\*
	Number of Variables (Columns): 199\\*
	Number of Entries (Rows): 2854
	\vspace{1.5em}
	
	\noindent Quick Summary: Similar to other state datasets with little distinguishing characteristics. \textit{There \emph{\textbf{is}} data specifically concerning banks or the financial sector in this dataset.}
	
	\paragraph{Data Specifics} (\guillemotleft~\guillemotright~indicates a break in data information, consult codebook):
	\vspace{1em}
	
	\begin{tabular}{r|l}
		001--003 & Locational Data\\
		004--044, 193--195 & Population Data\\
		045--046 & \guillemotleft~\guillemotright\\
		047--095, 172--179 & Agricultural Data\\
		096--122 & Manufacturing Data\\
		123--130 & Manfuacturing Data\\
		131--171 & Population Data by Nation\\
		180--183 & \guillemotleft~\guillemotright\\
		\textbf{184--192} & \textsc{ \textbf{Mortgage and Interest Data} }\\
		196--197 & Population Data by State\\
		198--199 & \guillemotleft~\guillemotright
	\end{tabular}
	\newpage

\section{DS19 --- 1890 Census}
	Data Level: State\\*
	Number of Variables (Columns): 259\\*
	Number of Entries (Rows): 50
	\vspace{1.5em}
	
	\noindent Quick Summary: Similar to previous datasets. \textit{There is no data specifically concerning banks or the financial sector.}
	
	\paragraph{Data Specifics} (\guillemotleft~\guillemotright~indicates a break in data information, consult codebook):
	\vspace{1em}
	
	\begin{tabular}{r|l}
		001--002 & Locational Data\\
		003--013 & Population Data\\
		014--121 & Population Data for White Population by State\\
		122--175 & Population Data for Color Population by State\\
		176--216 & Population Data by Nation\\
		217--242 &\guillemotleft~\guillemotright\\
		243--244 & Agricultural Data\\
		245--254 & Vital Statistics Data, Marriages\\
		255--256 & \guillemotleft~\guillemotright\\
		257--258 & Population Data by State\\
		259 & \guillemotleft~\guillemotright
	\end{tabular}
	\newpage

\section{DS20 --- 1900 Census}
	Data Level: State and National\\*
	Number of Variables (Columns): 191\\*
	Number of Entries (Rows): 2972
	\vspace{1.5em}
	
	\noindent Quick Summary: Similar to previous datasets. \textit{There is no data specifically concerning banks or the financial sector.}
	
	\paragraph{Data Specifics} (\guillemotleft~\guillemotright~indicates a break in data information, consult codebook):
	\vspace{1em}
	
	\begin{tabular}{r|l}
		001--003 & Locational Data\\
		004--022, 037--048, 054 & Population Data\\
		023--036, 049--053 & Literacy Rates\\
		055--058 & \guillemotleft~\guillemotright\\
		059--106 & Population Data by Nation\\
		107--132 & Manufacturing Data\\
		133--168 & Agricultural Data\\
		169--182 & \guillemotleft~\guillemotright\\
		183--187 & Population Data by Gender\\
		188--189 & Population Data by State\\
		190--191 & \guillemotleft~\guillemotright
	\end{tabular}
	\newpage

\section{DS21 --- 1900 Census}
	Data Level: State\\*
	Number of Variables (Columns): 20\\*
	Number of Entries (Rows): 55
	\vspace{1.5em}
	
	\noindent Quick Summary: Similar to previous datasets. \textit{There is no data specifically concerning banks or the financial sector.}
	
	\paragraph{Data Specifics} (\guillemotleft~\guillemotright~indicates a break in data information, consult codebook):
	\vspace{1em}
	
	\begin{tabular}{r|l}
		001--002 & Locational Data\\
		003--016 & Population Data by Gender\\
		017 & \guillemotleft~\guillemotright\\
		018--019 & Population Data by State\\
		020 & \guillemotleft~\guillemotright
	\end{tabular}
	\newpage

\section{DS22 --- 1910 Census}
	Data Level: County and State\\*
	Number of Variables (Columns): 195\\*
	Number of Entries (Rows): 3013
	\vspace{1.5em}
	
	\noindent Quick Summary: Data 004--069 is labeled as Population Data but due to data organization, the range also contains lieracy rates, race data, and education data. \textit{There is no data specifically concerning banks or the financial sector.}
	
	\paragraph{Data Specifics} (\guillemotleft~\guillemotright~indicates a break in data information, consult codebook):
	\vspace{1em}
	
	\begin{tabular}{r|l}
		001--003 & Locational Data\\
		004--069, 190--191 & Population Data\\
		070--071 & \guillemotleft~\guillemotright\\
		072--131 & Population Data by Nation\\
		132--176 & Agricultural Data\\
		177--189 & \guillemotleft~\guillemotright\\
		192--193 & Population Data by State
	\end{tabular}
	\newpage

\section{DS23 --- 1910 Census}
	Data Level: State\\*
	Number of Variables (Columns): 116\\*
	Number of Entries (Rows): 50
	\vspace{1.5em}
	
	\noindent Quick Summary: DS23 is almost entirely agricultural data. \textit{There is no data specifically concerning banks or the financial sector.}
	
	\paragraph{Data Specifics} (\guillemotleft~\guillemotright~indicates a break in data information, consult codebook):
	\vspace{1em}
	
	\begin{tabular}{r|l}
		001--003 & Locational Data\\
		004--013 & Population Data\\
		014--112 & Agricultural Data\\
		113--114 & Population Data by State
	\end{tabular}
	\newpage

\section{DS24 --- 1920 Census}
	Data Level: State and National\\*
	Number of Variables (Columns): 199\\*
	Number of Entries (Rows): 3132
	\vspace{1.5em}
	
	\noindent Quick Summary: Data 004--068 is labeled as Population Data but the range also contains literacy rates. \textit{There is no data specifically concerning banks or the financial sector.}
	
	\paragraph{Data Specifics} (\guillemotleft~\guillemotright~indicates a break in data information, consult codebook):
	\vspace{1em}
	
	\begin{tabular}{r|l}
		001--003 & Locational Data\\
		004--068 & Population Data\\
		069--070 & \guillemotleft~\guillemotright\\
		071--113 & Population Data for White Population by Nation\\
		114--121 & Manufacturing Data\\
		122--180 & Agricultural Data\\
		181--195 & \guillemotleft~\guillemotright\\
		196--197 & Population Data by State\\
		198--199 & \guillemotleft~\guillemotright
	\end{tabular}
	\newpage

\section{DS25 --- 1920 Census}
	Data Level: State\\*
	Number of Variables (Columns): 171\\*
	Number of Entries (Rows): 50
	\vspace{1.5em}
	
	\noindent Quick Summary: DS25 is almost entirely agricultural data. \textit{There is no data specifically concerning banks or the financial sector.}
	
	\paragraph{Data Specifics} (\guillemotleft~\guillemotright~indicates a break in data information, consult codebook):
	\vspace{1em}
	
	\begin{tabular}{r|l}
		001--003 & Locational Data\\
		004--031 & Special \textDelta~Data\\
		032--167 & Agricultural Data\\
		168--169 & Population Data by State\\
		170--171 & \guillemotleft~\guillemotright
	\end{tabular}
	\newpage

\section{DS43 --- 1870 Manufacturing Data}
	Data Level: State and National\\*
	Number of Variables (Columns): 22\\*
	Number of Entries (Rows): 2365
	\vspace{1.5em}
	
	\noindent Quick Summary: Similar to previous datasets, although note that DS43 is not based on census data. \textit{There is no data specifically concerning banks or the financial sector.}
	
	\paragraph{Data Specifics} (\guillemotleft~\guillemotright~indicates a break in data information, consult codebook):
	\vspace{1em}
	
	\begin{tabular}{r|l}
		001--003 & Locational Data\\
		004 & Population Data\\
		005--017 & Manufacturing Data\\
		018--020 & \guillemotleft~\guillemotright\\
		021--022 & Population Data by State
	\end{tabular}

\end{document}