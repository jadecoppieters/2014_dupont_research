\documentclass[12pt]{report}
\usepackage[utf8]{inputenc}
\usepackage[T1]{fontenc}
\usepackage{titling}
\usepackage{fullpage}
\usepackage{bold-extra}

\newcommand{\subtitle}[1]{%
    \posttitle{%
        \par\end{center}
        \begin{center}\large#1\end{center}
        \vskip0.5em}%
}

\title{\Huge \bf Document Summary}
\subtitle{George E. Barnett --- \textit{State Banks and Trust Companies\\ Since the Passage of the National-Bank Act}}
\date{July 12, 2014}
\author{Jade Coppieters}

\begin{document}

\maketitle
\tableofcontents
\newpage

\chapter{Minimum Capital Required for State Bank Incorporation in 1911 by Class Groupings}

\section{No Minimum/Business Incorporation States}
Of the states that allow incorporation of a state bank under general laws, 4 had no minimum capital requirements (\textit{listed below}). In these states, banks are incorporated under ``business incorporation law'', and the amount of capital required is left to the incorporating bank's discretion.\\

    \begin{enumerate}
        \item Arizona
        \item Arkansas
        \item South Carolina
        \item Tennessee
    \end{enumerate}

\section{Class 1 States/Uniform Requirements}
In some states, the minimum capital requirement is the same for all banks, regardless of location or size of business:\\

    \begin{enumerate}
        \item Georgia: \$15,000/\$25,000. While the minimum requirement is nominally \$25,000, there was no system in 1911 for demanding more than \$15,000.
        \item Indiana: \$25,000.
        \item Montana: \$20,000.
        \item New Jersey: \$50,000.
        \item New Mexico: \$30,000.
        \item Ohio: \$25,000.
        \item Virginia: \$10,000.
        \item West Virginia: \$25,000.
    \end{enumerate}

\section{Class 2 States/Population Grading}
In some states, the minimum capital requirement varies with the population of the town/city in which the bank is located:

    \begin{enumerate}
        \item Alabama:\\*
            \begin{tabular}{r|l}
                <2,500 Inhabitants & \$15,000\\
                2,500+ Inhabitants & \$25,000
            \end{tabular}
        \item Colorado:\\*
            \begin{tabular}{r|l}
                <5,000 Inhabitants & \$15,000\\
                5,000--10,000 Inhabitants & \$25,000\\
                10,000+ Inhabitants & \$30,000
            \end{tabular}
        \item Florida:\\*
            \begin{tabular}{r|l}
                <3,000 Inhabitants & \$15,000\\
                3,000--6,000 Inhabitants & \$25,000\\
                6,000+ Inhabitants & \$50,000
            \end{tabular}
        \item Idaho:\\*
            \begin{tabular}{r|l}
                <2,000 Inhabitants & \$10,000\\
                2,000--3,000 Inhabitants & \$20,000\\
                3,000--5,000 Inhabitants & \$25,000\\
                5,000--10,000 Inhabitants & \$30,000\\
                10,000--25,000 Inhabitants & \$50,000\\
                25,000+ Inhabitants & \$100,000
            \end{tabular}
        \item Iowa:\\*
            \begin{tabular}{r|l}
                <3,000 Inhabitants & \$25,000\\
                3,000+ Inhabitants & \$50,000
            \end{tabular}
        \item Illinois:\\*
            \begin{tabular}{r|l}
                <5,000 Inhabitants & \$25,000\\
                5,000--10,000 Inhabitants & \$50,000\\
                10,000--50,000 Inhabitants & \$100,000\\
                50,000+ Inhabitants & \$200,000
            \end{tabular}
        \item Kentucky:\\*
            \begin{tabular}{r|l}
               <50,000 Inhabitants & \$15,000\\
               50,000+ Inhabitants & \$100,000
            \end{tabular}
        \item Louisiana:\\*
            \begin{tabular}{r|l}
                <2,500 Inhabitants & \$10,000\\
                2,500--10,000 Inhabitants & \$30,000\\
                10,000--20,000 Inhabitants & \$50,000\\
                20,000+ Inhabitants & \$100,000
            \end{tabular}
        \item Maryland:\\*
            \begin{tabular}{r|l}
                <1,500 Inhabitants & \$10,000\\
                1,500--3,500 Inhabitants & \$20,000\\
                3,500--5,000 Inhabitants & \$25,000\\
                5,000--10,000 Inhabitants & \$30,000\\
                10,000--50,000 Inhabitants & \$50,000\\
                50,000--150,000 Inhabitants & \$100,000\\
                150,000+ Inhabitants & \$200,000
            \end{tabular}
        \item Michigan:\\*
            \begin{tabular}{r|l}
                <1,500 Inhabitants & \$20,000\\
                1,500--5,000 Inhabitants & \$25,000\\
                5,000--20,000 Inhabitants & \$50,000\\
                20,000--110,000 Inhabitants & \$100,000\\
                110,000+ Inhabitants & \$250,000
            \end{tabular}
        \item Minnesota:\\*
            \begin{tabular}{r|l}
                <1,000 Inhabitants & \$10,000\\
                1,000--1,500 Inhabitants & \$15,000\\
                1,500--2,000 Inhabitants & \$20,000\\
                2,000+ Inhabitants & \$25,000
            \end{tabular}
        \item New York:\\*
            \begin{tabular}{r|l}
                <2,000 Inhabitants & \$25,000\\
                2,000--30,000 Inhabitants & \$50,000\\
                30,000+ Inhabitants & \$100,000
            \end{tabular}
        \item North Dakota:\\*
            \begin{tabular}{r|l}
                <1,000 Inhabitants & \$10,000\\
                1,000--2,000 Inhabitants & \$20,000\\
                2,000--3,000 Inhabitants & \$30,000\\
                3,000--4,000 Inhabitants & \$35,000\\
                4,000--5,000 Inhabitants & \$40,000\\
                5,000+ Inhabitants & \$50,000
            \end{tabular}
        \item North Carolina:\\*
            \begin{tabular}{r|l}
                <2,000 Inhabitants & \$5,000\\
                2,000--5,000 Inhabitants & \$10,000\\
                5,000+ Inhabitants & \$25,000
            \end{tabular}
        \item Oregon:\\*
            \begin{tabular}{r|l}
                <1,000 Inhabitants & \$10,000\\
                1,000--2,000 Inhabitants & \$25,000\\
                2,000--5,000 Inhabitants & \$30,000\\
                5,000+ Inhabitants & \$50,000
            \end{tabular}
        \item Pennsylvania:\\*
            \begin{tabular}{r|l}
                <5,000 Inhabitants & \$25,000\\
                5,000+ Inhabitants & \$50,000
            \end{tabular}
        \item Utah:\\*
            \begin{tabular}{r|l}
                <3,500 Inhabitants & \$10,000\\
                3,500--10,000 Inhabitants & \$25,000\\
                10,000--50,000 Inhabitants & \$50,000\\
                50,000+ Inhabitants & \$100,000
            \end{tabular}
        \item Washington:\\*
            \begin{tabular}{r|l}
                <1,000 Inhabitants & \$15,000\\
                1,000--2,000 Inhabitants & \$20,000\\
                3,000--5,000 Inhabitants & \$25,000\\
                5,000--10,000 Inhabitants & \$30,000\\
                10,000--25,000 Inhabitants & \$50,000\\
                25,000--50,000 Inhabitants & \$75,000\\
                50,000+ Inhabitants & \$100,000
            \end{tabular}
        \item Wisconsin:\\*
            \begin{tabular}{r|l}
                <1,500 Inhabitants & \$10,000\\
                1,500--3,500 Inhabitants & \$20,000\\
                3,500--5,000 Inhabitants & \$25,000\\
                5,000--10,000 Inhabitants & \$30,000\\
                10,000+ Inhabitants & \$50,000
            \end{tabular}
        \item Wyoming:\\*
            \begin{tabular}{r|l}
                <2,000 Inhabitants & \$10,000\\
                2,000--4,000 Inhabitants & \$25,000\\
                4,000--6,000 Inhabitants & \$50,000\\
                6,000+ Inhabitants & \$100,000
            \end{tabular}
    \end{enumerate}

\section{Class 3 States/Size Grading}
In some states, the minimum capital requirement varies by the amount of business done by the bank (\textit{listed below}). In all states in this category, there is an additional requirement akin to Class 1 or Class 2.

    \begin{enumerate}
        \item California: \$25,000. Additional laws requiring capital be in a certain proportion to either deposits or loans.
        \item Kansas:\\*
            \begin{tabular}{r|l}
                <500 Inhabitants & \$10,000\\
                500--1,000 Inhabitants & \$15,000\\
                1,000--2,000 Inhabitants & \$20,000\\
                2,000--15,000 Inhabitants & \$25,000\\
                15,000+ Inhabitants & \$50,000
            \end{tabular}
        \item Nebraska:\\*
            \begin{tabular}{r|l}
                <100 Inhabitants & \$10,000\\
                100--500 Inhabitants & \$15,000\\
                500--1,000 Inhabitants & \$20,000\\
                1,000--2,000 Inhabitants & \$25,000\\
                2,000--5,000 Inhabitants & \$35,000\\
                5,000--25,000 Inhabitants & \$50,000\\
                25,000--100,000 Inhabitants & \$100,000\\
                100,000+ Inhabitants & \$200,000
            \end{tabular}
        \item Nevada:\\*
            \begin{tabular}{r|l}
                <100 Inhabitants & \$10,000\\
                100--500 Inhabitants & \$15,000\\
                500--1,000 Inhabitants & \$20,000\\
                1,000--2,000 Inhabitants & \$25,000\\
                2,000--5,000 Inhabitants & \$35,000\\
                5,000+ Inhabitants & \$50,000
            \end{tabular}
        \item Oklahoma:\\*
            \begin{tabular}{r|l}
                <500 Inhabitants & \$10,000\\
                500--1,500 Inhabitants & \$15,000\\
                1,500--6,000 Inhabitants & \$25,000\\
                6,000--20,000 Inhabitants & \$50,000\\
                20,000+ Inhabitants & \$100,000
            \end{tabular}
        \item Rhode Island: \$25,000. The Rhode Island Board of Bank Incorporation determines the minimum capital requirement for incorporation.
        \item South Dakota:\\*
            \begin{tabular}{r|l}
                <1,500 Inhabitants & \$10,000\\
                1,500--2,500 Inhabitants & \$15,000\\
                2,500--5,000 Inhabitants & \$25,000\\
                5,000+ Inhabitants & \$50,000
            \end{tabular}
        \item Texas:\\*
            \begin{tabular}{r|l}
                <2,500 Inhabitants & \$10,000\\
                2,500--10,000 Inhabitants & \$25,000\\
                10,000--20,000 Inhabitants & \$50,000\\
                20,000+ Inhabitants & \$100,000
            \end{tabular}
    \end{enumerate}
    
\section{Unclassed States}
Included are states which, for whatever reason, Barnett has not grouped in any of the classes.

    \begin{enumerate}
        \item Mississippi:\\*
            \begin{tabular}{r|l}
                <500 Inhabitants & \$10,000\\
                500+ Inhabitants & \$15,000
            \end{tabular}
    \end{enumerate}



\chapter{Minimum Capital Required for State Bank Incorporation in 1911 by Region}

States and Territories can then be grouped into two main groups:\\

\section{Eastern States}
Eastern States have a minimum capital requirement of \$25,000, with the exception of Maryland (\$10,000) and New Jersey (\$50,000).

\section{All Other States/Territories}
The minimum capital required for incorporation is largely \$10,000, with the following exceptions:

\begin{enumerate}
\item North Carolina: \$5,000
\item Florida and Kentucky: \$15,000
\item Montana and Michigan: \$20,000
\item Indiana, Ohio, Iowa, Illinois, and California: \$25,000
\item New Mexico: \$30,000
\end{enumerate}



\chapter{Historical Increases in Minimum Capital Requirements}

In a few states, the original banking laws premitted incorporation with a minimm capital of \$5,000 but then increased the requirement to \$10,000 (\textit{listed below}).

\begin{enumerate}
\item Kansas
\item Nebraska
\item North Dakota
\item South Dakota
\item Oklahoma
\end{enumerate}

Wisconsin permitted bank incorporation with \$5,000 through a state act in 1903 and then increased to \$10,000.\\

Barnett reports that in \emph{all} states with a minimum capital requirement of \$5,000 there were complaints that the requirement was not satisfactory.



\chapter{State Minimum Capital Requirements by Bank Comparison}

\section{State Requirement > National Requirement}
In a few states, the minimum capital requirement for state banks is greater than that for national banks.

\begin{enumerate}
\item New Jersey and New Mexico: In cities/towns with less than 3,000 inhabitants.
\item New York: In cities/towns with 2,000--3,000 inhabitants.
\end{enumerate}

\section{State Requirement = National Requirement}
In seven states the requirement capital is the same for both state and national banks -- but only in locations with less than 3,000 inhabitants.

\begin{enumerate}
\item California
\item Indiana
\item Iowa
\item Illinois
\item Ohio
\item Pennsylvania
\item West Virigina
\end{enumerate}

\section{State Requirement < National Requirement -- Later Exceeds}
In five states the required capital is smaller for state banks than national banks in smaller cities/towns but then exceeds the required capital for national banks by 3,000 inhabitants.

\begin{enumerate}
\item Lousiana
\item Nevada
\item North Dakota
\item Nebraska
\item Oregon
\end{enumerate}

\section{State Requirement < National Requirement -- Later Equals}
In eight states the required capital is smaller for state banks than national banks in smaller cities/towns but then equals the required capital for national banks by 3,000 inhabitants.

\begin{enumerate}
\item Alabama
\item Kansas
\item Michigan
\item Oklahoma
\item Minnesota
\item South Dakota
\item Texas
\item Wyoming
\end{enumerate}

\section{State Requirement < National Requirement}
In all the remaining states the required capital is smaller for state banks than national banks in all places under 3,000 inhabitants.

\begin{enumerate}
\item Colorado
\item Georgia
\item Florida
\item Idaho
\item Kentucky
\item Maryland
\item Mississippi
\item Missouri
\item Montana
\item North Carolina
\item Utah
\item Virginia
\item Washington
\item Wisconsin
\end{enumerate}

\end{document}